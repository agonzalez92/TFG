\begin{abstract}

En el presente documento se va realizar el estudio de la estabilidad de robots humanoides utilizando los métodos de péndulo invertido y Cart-table para calcular el Punto de Momento Cero (ZMP). Todos los experimentos han sido realizados con el robot humanoide TEO (Task Environment Operator) del grupo de investigación RoboticsLab, en el departamento de Ingeniería de Sistemas y Automática de la Universidad Carlos III de Madrid.
\setlength{\parskip}{5mm}

Los datos para realizar dicho estudio se obtenían de los sensores de fuerza-par situados en los tobillos, para el modelo de péndulo invertido, y la Unidad de Medida Inercial (IMU) situada en el centro del robot, para el modelo Cart-table, mediante datos de aceleraciones lineales. Para realizar los experimentos los programas utilizados fueron QtCreator, que es un IDE (entorno de desarrollo integrado) multiplataforma para programar en C++, y Python, que es un lenguaje de programación interpretado cuyo objetivo es simplificar el código del programa.

Dicho proyecto consiste en someter al robot a diferentes inclinaciones para ajustar su posición de equilibrio haciendo uso de los modelos anteriormente descritos, utilizando para ello unas mejoras que permiten ajustar la respuesta tanto en régimen permanente como en régimen transitorio de ambos modelos.

\end{abstract}