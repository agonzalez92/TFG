\section{Conclusiones y trabajos futuros}

Una vez realizadas todas las pruebas, se pueden sacar unas conclusiones que a continuación se expondrán, verificando si se han cumplido o no los objetivos propuestos en el presente proyecto, y se propondrán una serie de mejoras para trabajos futuros.

Uno de los objetivos iniciales fue ajustar el modelo que más se ajustara a la realidad. En este caso se ajustó el modelo LIPM mediante un controlador PD. Se puede concluir que mediante este controlador no se podía ajustar la respuesta global del modelo debido a que si se ajustaba la respuesta del régimen permanente, la oscilación inicial del régimen transitorio y su tiempo de estabilización eran muy grandes, y, en situaciones en las que el ZMP del robot estuviera situado al borde del polígono de soporte, ésto podría provocar que el robot perdiera el equilibrio. La situación inversa también se daba, es decir, se lograba reducir la amplitud de la oscilación inicial, pero entonces el régimen permanente presentaba un error respecto al ZMP de referencia.

Por lo tanto, se desarrolló un modelo a partir del LIPM, externo al actual proyecto, denominado DLIPM, y gracias a su experimentación, se verificó que cumplía con el siguiente objetivo del trabajo. En él se consiguió mejorar la respuesta global de los sensores, tanto en régimen permanente, reduciendo el error al mínimo con respecto al ZMP de referencia, como en régimen transitorio, reduciendo la amplitud de la oscilación inicial y el tiempo de estabilización del robot.

Por último, se puede confirmar el tercer objetivo, en el que se consiguió ajustar la respuesta del sensor IMU al ZMP de referencia mediante el ajuste de la altura del CoM del robot.

En cuanto a trabajos futuros, uno de los aspectos a mejorar del modelo DLIPM sería desarrollar un controlador PID para reducir tanto la amplitud de la oscilación inicial como reducir el tiempo de estabilización del sistema. Otra propuesta de mejora sería convertir el sistema subamortiguado en un sistema críticamente amortiguado, ya que así se podría evitar la oscilación inicial y asegurarse que cuando el ZMP estuviera al borde del polígono de soporte, se mantuviera el equilibrio del robot, reduciendo  también el tiempo de estabilización.