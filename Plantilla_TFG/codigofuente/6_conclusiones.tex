\section{Conclusiones y trabajos futuros}

Una vez realizadas todas las pruebas, se pueden sacar unas conclusiones que a continuación se expondrán, verificando si se han cumplido o no los objetivos propuestos en el presente proyecto, y se propondrán una serie de mejoras para trabajos futuros.

Como se ha descrito en este trabajo, éste consta de unas fases en las que se ha comprobado el correcto funcionamiento de los sensores, se han ajustado los modelos y se ha verificado su mejora mediante los diferentes experimentos. 

El objetivo inicial fue realizar un estudio del correcto funcionamiento de los sensores, tanto de los de fuerza-par como de la IMU, tanto en el robot como fuera de él mediante sensores externos que tenían las mismas características que los acoplados en el propio robot, y del programa que hace que dicha información pueda ser utilizada para el control dentro del ordenador. 

Una vez que ésto se verificó, se aplicaron los datos obtenidos de los sensores a los modelos estudiados a lo largo del presente trabajo. Se observó que al aplicar dicha información a los modelos, tanto LIPM como cart-table, éstos presentaban inexactitudes, debidas tanto a imperfecciones en el robot debidas a las tolerancias mecánicas como a la elasticidad de su estructura. Para intentar solucionar dichas imperfecciones e inexactitudes, se aplicó un controlador PD, pero los resultados no fueron del todo satisfactorios, por lo que se aplicó una mejora al modelo LIPM para mejorar la respuesta del controlador que permitió  acercar el punto de equilibrio del robot humanoide TEO al ZMP ideal. Dicha progreso hizo que el modelo LIPM pasase a denominarse DLIPM. En él se consiguió mejorar la respuesta global de los sensores, tanto en régimen permanente, reduciendo el desajuste al mínimo con respecto al ZMP de referencia, como en régimen transitorio, reduciendo la amplitud de la oscilación inicial y el tiempo de estabilización del robot.

Una vez que se verificó que la respuesta del controlador en el nuevo modelo DLIPM se incrementaba, ésta se equiparó en el modelo cart-table de la misma forma que para el modelo LIPM, ajustando su repuesta a partir de la premisa que los ZMP de ambos modelos debían coincidir, y como se observa en la figura \ref{figura516}, dicho objetivo se logró.

Como trabajos futuros se proponen las siguientes ideas:

\begin{itemize}

\item Como la mejora aplicada al modelo LIPM no conseguía reducir lo suficiente la amplitud de la oscilación inicial para asegurar que el robot mantendría el equilibrio en todo momento, sobre todo cuando el ZMP se sitúa al borde del polígono de soporte. Por tanto, se propone mejorar el controlador PD utilizado al principio.

\newpage

\item Otro posible trabajo futuro sería la utilización de modelos más complejos que permitieran un ajuste más fino, como puede ser el péndulo doble.

\item Para concluir, como las medidas de los sensores se ven afectadas por elementos físicos, se plantea mejorar las elasticidades mecánicas rigidizando la estructura del robot, una mejor situación de los sensores o la utilización de otro tipo de sensores, es decir, cambiar componentes del robot para evitar al máximo la influencia de esos elementos físicos en las medidas.

\end{itemize}
 


\afterpage{\null\newpage}
\newpage