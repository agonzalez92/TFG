\section{Conclusiones y trabajos futuros}

Una vez realizadas todas las pruebas, se pueden sacar unas conclusiones que a continuación se expondrán, verificando si se han cumplido o no los objetivos propuestos en el presente proyecto, y se propondrán una serie de mejoras para trabajos futuros.

Para iniciar el proyecto fue necesario un estudio del correcto funcionamiento de los sensores, tanto de los de fuerza-par como de la IMU. Una vez que ésto se verificó, se aplicaron los datos obtenidos de los sensores a los modelos estudiados a lo largo del presente trabajo. Se observó que al aplicar dicha información a los modelos, éstos presentabas errores que se intentaron corregir con un controlador PD, pero no fue suficiente, por lo que se aplicó una mejora que permitió resolver con éxito dichos errores y acercar el punto de equilibrio del robot humanoide TEO al ZMP ideal. Se consiguió mejorar la respuesta global de los sensores, tanto en régimen permanente, reduciendo el error al mínimo con respecto al ZMP de referencia, como en régimen transitorio, reduciendo la amplitud de la oscilación inicial y el tiempo de estabilización del robot.

Una vez que se implantó la mejora en el modelo de péndulo invertido simple, ésta se introdujo en el modelo cart-table, y como se observa en la figura \ref{figura516}, su respuesta se logró ajustar y ambos ZMP coincidieron.

Dicha mejora aplicada al modelo LIPM no conseguía reducir lo suficiente la amplitud de la oscilación incial para asegurar que el robot mantendría el equilibrio en todo momento, sobre todo cuando el ZMP se sitúa al borde del polígono de soporte. Por tanto, uno de los trabajos futuros que se proponen es desarrollar un controlador PID para reducir tanto la amplitud de la oscilación inicial como reducir el tiempo de estabilización del sistema. Otra propuesta de mejora sería convertir el sistema subamortiguado en un sistema críticamente amortiguado, ya que así se podría evitar la oscilación inicial y asegurarse que cuando el ZMP estuviera al borde del polígono de soporte, se mantuviera el equilibrio del robot.


\afterpage{\null\newpage}
\newpage