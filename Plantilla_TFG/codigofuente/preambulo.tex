\usepackage[utf8]{inputenc}
\usepackage[spanish]{babel}
\usepackage{amsmath}
\usepackage{amsfonts}
\usepackage{amssymb}
\usepackage{multirow}
\usepackage{blindtext}
\usepackage{enumitem}
%\usepackage{makeidx}
\usepackage{graphicx}
\usepackage{fancyhdr}
\usepackage{hyperref}
\usepackage{float}
\usepackage{color}
\usepackage{afterpage}
\usepackage{blindtext}
\definecolor{azulUC3M}{RGB}{0,0,102}
\definecolor{gray97}{gray}{.97}
\definecolor{gray75}{gray}{.75}
\definecolor{gray45}{gray}{.45}
\usepackage{listings}
\usepackage{parskip}
\usepackage{afterpage}
\usepackage{amssymb,amsfonts,latexsym,cancel}
\usepackage{float}
\usepackage{subfigure}
\lstset{ frame=Ltb,
     framerule=0pt,
     aboveskip=0.5cm,
     framextopmargin=3pt,
     framexbottommargin=3pt,
     framexleftmargin=0.4cm,
     framesep=0pt,
     rulesep=.4pt,
     backgroundcolor=\color{gray97},
     rulesepcolor=\color{black},
     %
     stringstyle=\ttfamily,
     showstringspaces = false,
     basicstyle=\small\ttfamily,
     commentstyle=\color{gray45},
     keywordstyle=\bfseries,
     %
     numbers=left,
     numbersep=15pt,
     numberstyle=\tiny,
     numberfirstline = false,
     breaklines=true,
   }
 
% minimizar fragmentado de listados
\lstnewenvironment{listing}[1][]
   {\lstset{#1}\pagebreak[0]}{\pagebreak[0]}
 
\lstdefinestyle{consola}
   {basicstyle=\scriptsize\bf\ttfamily,
    backgroundcolor=\color{gray75},
   }
 
\lstdefinestyle{C}
   {language=C,
   }
\usepackage[top=2cm]{geometry}
\pretolerance=2000
\tolerance=3000
\usepackage{verbatim} % comentarios
\renewcommand{\thefigure}{\thesection.\arabic{figure}}%renombra las figuras por capítulos y secciones
\usepackage{chngcntr}
\counterwithin{figure}{section}%estas tres últimas líneas para poner el número de la figura como sección.númeroimagen
%%\renewcommand\thefigure{\getCurrentSectionNumber-\arabic{figure}}
\counterwithin{table}{section}
\usepackage[nottoc]{tocbibind}
\usepackage{glossaries} %para acrónimos
\makeglossaries
\usepackage{afterpage} %para páginas en blanco
\usepackage{colortbl} %Para colores en las tablas
\usepackage{multirow,array}%para las tablas
%\usepackage[table,xcdraw]{xcolor}
\usepackage{ulem}%para que cuando se utilice el comando \emph{•} la palabra entre llaves esté subrayada. si lo quitamos la palabra estría en cursiva
\usepackage{eurosym} % para el euro
\renewcommand{\theequation}{\thesection.\arabic{equation}} %para poner el número del capítulo en las ecuaciones
\counterwithin{equation}{section} %para reiniciar el número de ecuación en cada capítulo
\usepackage{cite} %para referencias a la bibliografía
\usepackage{hyperref}
\usepackage{pdfpages} %para incluir documentos en pdf
\usepackage{appendix}
