\begin{thebibliography}{99} %%para poder incluir hasta 999 citas
%%Iniciales nombre Apellidos autor, "Título del artículo web/post", Título del web/blog en cursiva, Fecha de publicación. [En línea]. Disponible en: URL del recurso.
%%G. Peris Ripollés, "Acertando quinielas con redes neuronales", Naukas, 09-12-2015. [En línea]. Disponible en: http://naukas.com/2015/12/09/acertando-quinielas-redes-neuronales/.
\bibitem{ref1} Milicua Ainhoa, "Las 6 leyes de la robótica de la Unión Europea", Blogthinking, 21-06-2017. [En línea]. Disponible en: \url{https://blogthinkbig.com/las-6-leyes-de-la-robotica-de-la-union-europea}. 

\bibitem{ref2} Wikipedia, "Robótica", Wikipedia, 2008. [En línea]. Disponible en: \url{https://es.wikipedia.org/wiki/Robótica}.

\bibitem{ref3} Wikipedia, "Robot", Wikipedia, 2007. [En línea]. Disponible en: \url{https://es.wikipedia.org/wiki/Robot}.

\bibitem{ref4} International Organization for Standardization, "Robots and robotic devices -- Vocabulary", ISO, 03-2012. [En línea]. Disponible en: \url{https://www.iso.org/standard/55890.html?browse=tc}.

\bibitem{ref5} Asociación Española de Robótica y Automatización, AER-Automation. [En línea]. Disponible en: \url{https://www.aer-automation.com/aer-atp/robotica-industrial-y-de-servicio/}. Acceso: junio 2018.

\bibitem{ref6} K. Sanchez Madriz, "Clasificación de los robots según su arquitectura", El avance de la Robótica, 31-10-2011. [En línea]. Disponible en: \url{https://sites.google.com/site/elavancedelarobotica/clasificacion-de-los-robots/clasificacion-de-los-robots-su}.

\bibitem{ref7} Robots and Androids, "The History of Robots", Robots and Androids. [En línea]. Disponible en: \url{http://www.robots-and-androids.com/history-of-robots.html}. Acceso: junio 2018.

\bibitem{ref8} Md Akhtaruzzaman y A.A. Shafie, " Evolution of Humanoid Robot and contribution of various countries in advancing the research and development of the platform", presentada en Control Automation and Systems (ICCAS), Korea, 27-30 oct. 2010. [En línea]. Disponible en: \url{https://ieeexplore.ieee.org/document/5669646/}. Acceso: julio 2018.

\bibitem{ref9} T. Martín y A. Serrano, "Dinámica de sistemas. Centro de masas", Universidad Politécnica de Madrid. [En línea]. Disponible en: \url{http://www2.montes.upm.es/dptos/digfa/cfisica/dinamsist/cdm.html}.

\bibitem{ref10} S. Kajita, H. Hirukawa, K. Harada y K. Yokoi, "Introduction to Humanoid Robotics", Springer-Verlag Berlin Heidelberg, 2014. [En línea]. Disponible en: \url{https://link.springer.com/book/10.1007\%2F978-3-642-54536-8}.

\bibitem{ref11} S. Fernández, "Locomoción bípeda del robot humanoide NAO", Proyecto final de carrera, Universidad Politécnica de Catalunya, Cataluña, España, 2009. [En línea]. Disponible en: \url{https://upcommons.upc.edu/handle/2099.1/9115}. Acceso: julio 2018.

\bibitem{ref12} Xsens Technologies B.V., "MTi Miniature Attitude and Heading Reference System", Xsens Technologies B.V., Holanda, 2009.

\bibitem{ref13} Xsens Technologies B.V., "MTi and MTx User Manual and Technical Documentation", Xsens Technologies B.V., Holanda, Informe técnico MT0100P , 2006.

\bibitem{ref14} G. Metta, P. Fitzpatrick y L. Natale, "YARP: Yet another robot platform", \textsl{International Journal of Advanced Robotic Systems}, vol. 3, n.º 1, pp. 43-48, marzo 2006. [En línea]. Disponible en: \url{http://journals.sagepub.com/doi/abs/10.5772/5761}. Acceso: julio 2018.

\bibitem{ref15} S. Kajita, F. Kanehiro, K. Kaneko, K. Yok oi y H. Hirukawa, "The 3D Linear Inverted Pendulum Mode: A simple modeling for a biped walking pattern generation", presentada en International Conference on Intelligent Robots and Systems, Maui, Hawaii, USA, Oct. 29 - Nov. 03, 2001. [En línea]. Disponible en: \url{https://pdfs.semanticscholar.org/6a31/6e0d44e35a55c41a442b3f0d0eb1f9d4d0ca.pdf}. Acceso: julio 2018.

\bibitem{ref16} M. Vukobratović y B. Borovac, "Zero Moment Point - Thirty five years of its life", \textit{International Journal of Humanoid Robotics}, vol. 1, n.º 1, pp. 157–173, 2004. [En línea]. Disponible en: \url{https://www.researchgate.net/publication/220065796_Zero-Moment_Point_-_Thirty_Five_Years_of_its_Life} . Acceso: julio 2018.

\bibitem{ref17} D.A. Winter , F. Prince, J. S. Frank, C. Powell and K. F. Zabjek, "Unified theory regarding A/P and M/L balance in quiet stance", \textit{Journal of neurophysiology}, vol. 75, n.º 6, pp. 2334-2343, julio 1996. [En línea]. Disponible en: \url{https://www.researchgate.net/publication/14409014_Unified_theory_regarding_AP_and_ML_balance_in_quiet_stance}. Acceso: agosto 2018.

\bibitem{ref18} D.N. Nenchev y A. Nishio, " Experimental Validation of Ankle and Hip Strategies for Balance Recovery with a Biped Subjected to an Impact", presentada en International Conference on Intelligent Robots and Systems, San Diego, CA, USA, Oct. 29 - Nov. 2, 2007. [En línea]. Disponible en: \url{https://www.researchgate.net/publication/4296918_Experimental_validation_of_ankle_and_hip_strategies_for_balance_recovery_with_a_biped_subjected_to_an_impact}.

\bibitem{ref19} M. Vukobratović, "Humanoid Robotics - Past, Present State, Future", presentada en $4^{th}$ Serbian-Hungarian Joint Symposium on Intelligent Systems, Belgrado, 2006. [En línea]. Disponible en: \url{http://conf.uni-obuda.hu/sisy2006/1_Vuk.pdf}.

\bibitem{ref20} S. Kajita y B. Espiau, "16. Legged Robot", en B. Siciliano y O. Khatib (Eds.) \textit{Springer Handbook of Robotics}. Berlin, Alemania: Springer-Verlag, 2008, pp. 361-389.  

\bibitem{ref21} S. Martínez, J.M. García-Haro, J.G. Victores, A. Jardon y C. Balaguer, " Experimental Robot Model Adjustments Based on Force-Torque Sensor Information", \textit{Sensors}, vol. 18, n.º 3, art. nº 836, marzo 2018. [En línea]. Disponible en: \url{http://www.mdpi.com/1424-8220/18/3/836}.

\bibitem{ref22} J. Lorente, " Equilibrium control for humanoid robot TEO by inertial perception", Trabajo fin de máster, Dpto. de Ingeniería de Sistemas y Automática, Universidad Carlos III de Madrid, Madrid, España, 2016.

\bibitem{ref23} M. D. Pinel, "Balance control of humanoid robot TEO using force/torque sensors", Trabajo fin de máster, Dpto. de Ingeniería de Sistemas y Automática, Universidad Carlos III de Madrid, Madrid, España, 2016.

\bibitem{ref24} M. H. P. Dekker, "Zero-Moment Point method for stable biped walking", University of Technology, Eindhoven, Informe Técnico 2009.072, Julio 2009.

\bibitem{ref25} M. B. Popovic, A. Goswani y H. Herr, "Ground Reference Points in Legged Locomotion: Definitions, Biological Trajectories and Control Implications", \textit{The International Journal of Robotics Research}, ene. 2005. 

\bibitem{ref26} S. Martínez, "Human Inspired Humanoid Robot Control Architecture", Tesis doctoral, Dpto. de Ingeniería de Sistemas y Automática, Universidad Carlos III de Madrid, Leganés, España, 2012. [En línea]. Disponible en: \url{https://e-archivo.uc3m.es/handle/10016/16422}.

\bibitem{ref27} AENOR, "UNE-EN ISO 10218-1:2012", \textit{AENOR}, 2012. [En línea]. Disponible en: \url{https://www.aenor.com/normas-y-libros/buscador-de-normas/UNE?c=N0049289}.

\bibitem{ref28} AENOR, "UNE-EN ISO 10218-2:2011", \textit{AENOR}, 2011. [En línea]. Disponible en: \url{https://www.aenor.com/normas-y-libros/buscador-de-normas/UNE?c=N0048668}.

%\bibitem{ref29} R. Álvarez, "El asombroso robot 'Atlas' de Boston Dynamics aprende a saltar y ahora es más aterrador (y espectacular) que antes", \textit{Xataka}, 17-11-2017. [En línea]. Disponible en: \url{https://www.xataka.com/robotica-e-ia/el-asombroso-robot-atlas-de-boston-dymanics-aprende-a-saltar-y-ahora-es-mas-aterrador-y-espectacular-que-antes}.

\bibitem{ref29} "Mars Pathfinder Rover: Sojourner", \textit{Smithsonian: National Air and Space Museum}, 2011. [En línea]. Disponible en: \url{https://airandspace.si.edu/multimedia-gallery/web12070-2011640jpg}.

\bibitem{ref30} Omega Engineering, "Acelerómetro", \textit{Omega}. [En línea]. Disponible en: \url{https://es.omega.com/prodinfo/acelerometro.html}.

\bibitem{ref31} Wikipedia, "Giróscopo", Wikipedia, 06-10-2018. [En línea]. Disponible en: \url{https://es.wikipedia.org/wiki/Giróscopo}.

\bibitem{ref32} G. Metta, G. Sandini, D. Vernon, L. Natale y F. Nori, "The iCub humanoid robot: an open platform for research in embodied cognition", presentada en Advanced Robotics and its Social Impacts, Half-Moon Bay, CA, USA, 02-04 oct., 2011. [En línea]. Disponible en: \url{https://ieeexplore.ieee.org/document/6301975/}.

\bibitem{ref33} S. Kajita et al., "Biped Walking Pattern Generation by using Preview Control of Zero-Moment Point", presentada en International Conference on Robotics and Automation, Taipei, Taiwan, 14-19 sep., 2003. [En línea]. Disponible en: \url{https://ieeexplore.ieee.org/document/1241826/}.

\bibitem{ref34} \textit{Página web de Waseda University}. [En línea]. Disponible en: \url{http://www.humanoid.waseda.ac.jp/}.

\bibitem{ref35} \textit{Página web de Boston Dynamics}. [En línea]. Disponible en: \url{https://www.bostondynamics.com/}.

\bibitem{ref36} \textit{Página web de Anybotics}. [En línea]. Disponible en: \url{https://www.anybotics.com/}.

\bibitem{ref37} \textit{Página web de ABB}. [En línea]. Disponible en: \url{https://new.abb.com/}.

\bibitem{ref38} S.J. Van Dalen, "A Linear Inverted Pendulum Walk Implemented on TUlip", Trabajo fin de máster, Dpto. de Ingeniería Mecánica, University of Technology, Eindhoven, Países Bajos, 2012. [En línea]. Disponible en: \url{https://www.techunited.nl/media/files/humanoid/SwanVanDalen_GRAD2012_A_Linear_Inverted_Pendulum_Walk_Implemented_on_TUlip.pdf}.

\bibitem{ref39} "Managing Procurement in a Digital World", \textit{Sourcing Guy}, 16-09-2015. [En línea]. Disponible en: \url{https://sourcingguy.wordpress.com/2015/09/16/managing-procurement-in-a-digital-world/}.

\end{thebibliography} 

\afterpage{\null\newpage}
\newpage